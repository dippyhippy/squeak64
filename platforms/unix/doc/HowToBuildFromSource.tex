\documentclass{article}

\usepackage{html,makeidx,color,alltt}

\parindent 0pt
\parskip 5pt plus 3pt minus 2pt

%begin{latexonly}
\usepackage{url}
\newcommand{\cmd}{\url}
%end{latexonly}
\begin{htmlonly}
\newcommand{\cmd}{\texttt}
\end{htmlonly}

% The ridiculous stuff (like alltt inside textcolor, etc.) is necessary
% to achieve acceptable output simultaneously in .dvi, .html and .txt.

\newenvironment{note}{\begin{quote}\color{red} Note: }{\end{quote}}
\newenvironment{pre}{\begin{quote}\begin{alltt}}{\end{alltt}\end{quote}}

\makeindex

\begin{document}
\latexhtml
  {\title{Building Unix Squeak ($>=$ 3.2) from source}}%
  {\title{Building Unix Squeak (>= 3.2) from source}}
\author{Ian Piumarta\\
\htmladdnormallink{\texttt{<ian.piumarta@inria.fr>}}{mailto:ian.piumarta@inria.fr}}
\date{\small Last edited: 2002-06-05 05:22:02 by piumarta on emilia.inria.fr
\\Translated to .ps/.pdf/.html/.txt: \today}
\maketitle
\tableofcontents

% \section*{Availability}
% 
% The latest stable release of Unix Squeak is available from:
% \begin{itemize}
% \item\htmladdnormallink%
%   {http://www-sor.inria.fr/~piumarta/squeak/index.html#stable}%
%   {http://www-sor.inria.fr/~piumarta/squeak/index.html#stable}
% \end{itemize}
% The latest development release of Unix Squeak is available from:
% \begin{itemize}
% \item\htmladdnormallink%
%   {http://www-sor.inria.fr/~piumarta/squeak/index.html#devel}%
%   {http://www-sor.inria.fr/~piumarta/squeak/index.html#devel}
% \end{itemize}
% The latest version of this document is available from:
% \begin{itemize}
% \item\htmladdnormallink%
%   {http://www-sor.inria.fr/~piumarta/squeak/unix/devel/build-html/build.html}%
%   {http://www-sor.inria.fr/~piumarta/squeak/unix/devel/build-html/build.html}
% \end{itemize}
% \begin{htmlonly}
% This document is also available in the following formats:
% \htmladdnormallink{PostScript}{../build.ps},
% \htmladdnormallink{PDF}{../build.pdf}, and
% \htmladdnormallink{text}{../build.txt}.
% \end{htmlonly}

\section{\texttt{configure} -- \texttt{build} -- \texttt{install}}

Unix Squeak is built using the (almost) universal ``\cmd{configure;
make; make install}''.  If you haven't come across this before, read on\ldots

Create a build\index{build directory!creating}
directory (which we will call `\cmd{blddir}' from now on)
and then `\cmd{cd}' to it:
\html{\textcolor{blue}}{\begin{alltt}\color{blue}
  \$ mkdir blddir
  \$ cd blddir
\end{alltt}}

A convenient place is just next to the \cmd{platforms} directory,
like this:
\html{\textcolor{blue}}{\begin{alltt}\color{blue}
  \$ cd squeak
  \$ ls
  platforms src ...
  \$ mkdir bld
  \$ cd bld
\end{alltt}}

Create the build\index{build directory!configuring}
environment by running the script \cmd{configure} which
lives in the \cmd{platforms/unix/config} directory.

\begin{note}
The \cmd{configure} script accepts lots of options.  To see a list of
them, run: `\verb|configure --help|'
\end{note}

Assuming you've created the \cmd{blddir} next to \cmd{platforms},
this would be:
\html{\textcolor{blue}}{\begin{alltt}\color{blue}
  \$ ../platforms/unix/config/configure
\end{alltt}}

\begin{note}
This assumes that the VMMaker sources are in `\cmd{../src}'.  However,
since the Unix Squeak support code is independent of the image version
from which VMMaker generated the interpreter/plugin sources, it is
possible that your source distribution comes with more than one
\cmd{src} directory (corresponding to more than one image version used
to generate the sources).  In such cases you will have to tell
\cmd{configure} which source version to use, via the `\html{-}\cmd{--with-src}'
option.  For example, if there
are two source directories called \cmd{src-3.2gamma-4857} and
\cmd{src-3.3.alpha-4881} then you would use \emph{one} of the
following commands:
\html{\textcolor{red}}{\begin{alltt}\color{red}
  \$ .../configure \html{-}--with-src=src-3.2gamma-4857
        \emph{or}
  \$ .../configure \html{-}--with-src=src-3.3alpha-4881
\end{alltt}}
\end{note}

\noindent
Build the VM and plugins by running \cmd{make}:
\html{\textcolor{blue}}{\begin{alltt}\color{blue}
  \$ make
\end{alltt}}

\begin{note}
If you want to build just the VM (without external plugins) or just
the external plugins (without the VM) then you can use:
`\verb|make squeak|' or `\verb|make plugins|' respectively.
\end{note}

Finally install the VM, plugins and manual pages:
\html{\textcolor{blue}}{\begin{alltt}\color{blue}
  \$ su root
  \$ make install
\end{alltt}}


\section{Generating your own VM and plugin sources}

Generating your own VM/plugin sources might be necessary for various reasons:
\begin{itemize}
\item you want to change the mix of internal vs. external plugins
\item you want to remove some plugins from the VM that you will never use
\item you've pulled in some updates that modify the Interpreter or plugins
\item you've filed-in (or written) a whole new plugin
\item etc\ldots
\end{itemize}

Version 3.2 (and later) of Unix Squeak use
\htmladdnormallink{VMMaker}{http://minnow.cc.gatech.edu/squeak/2105}%
\index{VMMaker!reference}
to generate the core interpreter and plugin sources.

Start Squeak in the top-level directory (the one containing the
\cmd{platforms} directory); for example:
\html{\textcolor{blue}}{\begin{alltt}\color{blue}
  \$ ls
  src platforms ...
  \$ squeak MyCoolPlugin.image
\end{alltt}}

Open a VMMakerTool and modify the setup to your liking.

\begin{note}
The VMMaker configuration used to build the distributions of Unix
Squeak is available in
\cmd{platforms/unix/config/VMMaker.config}.\index{VMMaker!configuration file}
\end{note}

Then click on the relevant ``\textsf{generate ...}'' button.  You can now
`\cmd{configure; make; make install}' in your \cmd{blddir} (as
described above).

\begin{note}
You only need to run
\cmd{configure}\index{config.status@\texttt{config.status}!versus \texttt{configure}}
\textbf{\underline{once}} for a given
\cmd{blddir} (on the same host).
If you modify the choice of plugins
(or change whether they're internal/external)
then you can update the build environment by running the
\cmd{config.status}\index{config.status@\texttt{config.status}}
script in the \cmd{bldddir}, like this:
\textcolor{red}{\begin{alltt}
  \$ squeak MyCoolPlugin.image
  ... generate new sources ...
  \$ cd blddir
  \$ ./config.status
  \$ make
\end{alltt}}
This is \emph{much} faster than running \cmd{configure} all over again.
(In fact, \cmd{make} should detect any changes to the plugin configuration
and re-run \cmd{config.status} for you automatically.)
\end{note}

\begin{note}
`\cmd{configure}' doesn't actually create any files.  The last thing it
does is run `\cmd{config.status}' to create the configured \emph{file}s
in \cmd{blddir} from the corresponding \emph{file.in}s in the
\cmd{unix/config} directory.  So in the remainder of this document the
phrase `during configuration' means \emph{either} when running
`\cmd{configure}' for the first time \emph{or} running
`\cmd{config.status}' to update an already \cmd{configure}d build
environment.
\end{note}


\section{Adding your own plugins}

\begin{note}
This section is intended primarily for plugin developers.
\end{note}

If your plugin\index{plugin!adding your own}
 requires no platform-specific tweaks then there's
nothing for you to do.  \cmd{configure} (and \cmd{config.status})
will provide a default \cmd{Makefile} for it that should work.  If your
plugin requires only platform-independent tweaks (and/or additional
hand-written code) then these go in \cmd{platforms/Cross/plugins}, and
there's nothing for you to do (in Unixland).

On the other hand, if you require special \cmd{configure} tests or
additional declarations/rules in your plugin's \cmd{Makefile} then you
need to specify them explicitly.

\begin{note}
Unix Squeak subscribes to the following philopsophy:

\emph{Absolutely everything that is specific\index{Unix-specific files}
 to Unix (sources,
headers, \cmd{configure} and \cmd{Makefile} extensions, etc.) lives
under \cmd{platforms/unix}.}

In other words: there is not (nor aught there be) \emph{any}
Unix-related information under the \cmd{platforms/Cross}
directory.  (Unix Squeak is entirely encapsulated
under \cmd{platforms/unix} and is utterly immune to
``random junk'' elsewhere in the \cmd{platforms} tree.)
\end{note}

First you must create a new directory\index{plugin!Unix-specific directory}
under \cmd{platforms/unix/plugins} named after your plugin.  This directory
will hold the files describing the additional configuration checks
and/or \cmd{Makefile} contents.  For example, if your plugin is
called ``MyCoolPlugin'' then
\html{\textcolor{blue}}{\begin{alltt}\color{blue}
  \$ mkdir platforms/unix/plugins/MyCoolPlugin
\end{alltt}}

would be the thing to do.  (The following sections will refer to this
directory as \cmd{platdep} since the full path is quite a mouthful of
typing for my lazy fingers.)


\subsection{Plugin-specific configuration}

Your plugin can ask \cmd{configure}\index{plugin!configuring}
to run additional tests (and to
set additional variables in its output files) simply by including a
file called \cmd{acinclude.m4}\index{acinclude.m4@\texttt{acinclude.m4}}
in it's \cmd{platdep} directory.

\begin{note}
The \cmd{configure}\index{configure@\texttt{configure}}
script is `compiled' from several other files.  If
you create a `\cmd{platdep./acinclude.m4}' file then you \emph{must}
`recompile'\index{configure@\texttt{configure}!recreating}
\cmd{configure}.  You can do this by `\cmd{cd}'ing to
\cmd{unix/config} and running `\cmd{make}', or (if you have GNU
\cmd{make}) from the \cmd{blddir} like this:
\html{\textcolor{red}}{\begin{alltt}\color{red}
  \$ make -C ../platforms/unix/config
\end{alltt}}
\end{note}

In addition to the usual \cmd{autoconf} macros, the following
macros\index{configure@\texttt{configure}!macros for plugins}
are available specifically for Squeak plugins to use:

\subsubsection{\texttt{AC\_PLUGIN\_CHECK\_LIB(\emph{lib},\emph{func})}}
\index{AC\_PLUGIN\_CHECK_LIB@\texttt{AC\_PLUGIN\_CHECK\_LIB}}

This is similar to the \cmd{autoconf} `\cmd{AC_CHECK_LIB}' macro.

\emph{func} is the name of a function required by the plugin, defined
in the external (system) library \emph{lib}.  The macro checks that
the library is available (via `\texttt{-l\emph{lib}}') and then adds it
to the list of libraries required by the plugin (see the explanation
of \texttt{[plibs]} in Section~\ref{sec:plibs} for a description of
how library dependencies for plugins are handled).

If \emph{func} cannot be found in \emph{lib} then the plugin will be
disabled and a message to that effect printed during configuration.
(The VM can still be built, \emph{without} rerunning VMMaker or
reconfiguring, and the plugin will simply be ommitted from it.)


\subsubsection{\texttt{AC\_PLUGIN\_DEFINE\_UNQUOTED(\emph{keyword},\emph{text})}}

This\index{AC\_PLUGIN\_DEFINE_UNQUOTED@\texttt{AC\_PLUGIN\_DEFINE\_UNQUOTED}}
is similar to the \cmd{autoconf} `\cmd{AC_DEFINE_UNQUOTED}' macro.

\emph{keyword} is a \cmd{Makefile} keyword (usually of the form
`[\emph{name}]') and \emph{text} is arbitrary text to be
associated with it.  Calling this macro causes \cmd{mkmf} to
substitute \emph{text} for all occurrences of \emph{keyword} in the
\cmd{Makefile} generated for the plugin.

\subsubsection{Plugin-specific variables}

The following variables are also set during the execution of a
plugin-specific \cmd{acinclude.m4}:

\begin{itemize}
\item[] \texttt{\$\{plugin\}} is the name of the plugin;
\item[] \texttt{\$\{topdir\}} is the path to the top-level directory (containing
  \cmd{platforms});
\item[] \texttt{\$\{vmmdir\}} is the path to the VMMaker `\cmd{src}' directory.
\end{itemize}


\subsection{Plugin-specific \texttt{Makefile} declarations and rules}

Three\index{plugin!\texttt{Makefile}}
mechanisms are avilable for this:
\begin{enumerate}
\item scanning additional dirrectories for sources and headers;
\item including a few additional lines into the default \cmd{Makefile};
  and 
\item replacing entirely the default \cmd{Makefile} with a
  hand-written one.
\end{enumerate}
(The last option isn't as scary as it might sound:\@ read on\ldots)


\subsubsection{The anatomy of a plugin's \texttt{Makefile}}

Before proceeding, let's take a minute to understand how Unix Squeak
compiles and links files in its default \cmd{Makefile} for plugins.
The default
\cmd{Makefile}\index{Makefile@\texttt{Makefile}}\index{plugin!\texttt{Makefile} anatomy}
\latexhtml{is shown in Figure~\ref{fig:mf-default}.}{looks like this:}

%begin{latexonly}
\begin{figure}[ht]
%end{latexonly}
\begin{quote}\begin{quote}
\html{\textcolor{black}}{\begin{alltt}
# default Makefile for Unix Squeak plugins

[make_cfg]
[make_plg]

XINCLUDES       = [includes]
OBJS            = [targets]
TARGET          = [target]
PLIBS           = [plibs]

[make_inc]

\$(TARGET) : \$(OBJS) Makefile
        \$(LINK) \$(TARGET) \$(OBJS) \$(PLIBS)

[make_targets]

.force :
\end{alltt}}\end{quote}\end{quote}
%begin{latexonly}
\caption{Default \texttt{Makefile} ``template'' for plugins.\label{fig:mf-default}}
\end{figure}
%end{latexonly}

\begin{note}
The keywords\index{Makefile@\texttt{Makefile}!keyword substitution}
appearing between `\texttt{[} square brackets \texttt{]}'
are substituted during configuration by a preprocessor called
`\cmd{mkmf}'\index{mkmf@\texttt{mkmf}}
according to the kind of plugin (internal/external) being built.
\end{note}

\texttt{\textbf{[make\_cfg]}}\index{[make\_cfg]@\texttt{[make\_cfg]}}
\index{Makefile keywords@\texttt{Makefile} keywords!\texttt{[make\_cfg]}}
is the configured variable section.  It
contains the platform-specific information gleaned by \cmd{configure}
while it was figuring out which compiler you have, what flags your
linker needs, where to install stuff, and so on.

\texttt{\textbf{[make\_plg]}}\index{[make\_plg]@\texttt{[make\_plg]}}
\index{Makefile keywords@\texttt{Makefile} keywords!\texttt{[make\_plg]}}
contains a handful of definitions which
depend on whether the plugin is being compiled as internal or
external:
\begin{quote}\begin{tabular}{@{\ttfamily}l@{~~~}l}
o       & the extension for object files \\
a       & the extension for plugins \\
COMPILE & the command to compile a source file into an object file \\
LINK    & the command to link one or more object files into a plugin \\
\end{tabular}\end{quote}

For internal plugins: \texttt{\$o} is `\cmd{.o}' and \texttt{\$a} is
`\cmd{.a}'.  \texttt{\$(COMPILE)} is the C compiler
(`\texttt{\$(CC)~\ldots~-o}', so the first thing after the command
\emph{must} be the output filename) and \texttt{\$(LINK)} is archiver
(`\texttt{ar -rc}', again requiring the output file to follow
immediately).  Note that internal plugins are built as `\cmd{ar}'
archives before being linked into the final binary.

For external plugins: \texttt{\$o} is `\cmd{.lo}', \texttt{\$a} is
`\cmd{.la}', and \texttt{\$(COMPILE)} and \texttt{\$(LINK)} are
invocations of `\cmd{libtool}' to create position-independent objects and
shared libraries (with a `\cmd{-o}' appearing right at the end, so the
first thing after the command \emph{must} be the output filename).

\texttt{\textbf{[includes]}}\index{[includes]@\texttt{[includes]}}
\index{Makefile keywords@\texttt{Makefile} keywords!\texttt{[includes]}}
is a list of~`\cmd{-I\emph{dir}}'
compiler flags, one for each of the
directories\index{mkmf@\texttt{mkmf}!default header directories}
\textcolor{black}{\begin{alltt}
  src/plugins/\emph{name}
  src/vm/intplugins/\emph{name}
  platforms/Cross/plugins/\emph{name}
  platforms/unix/plugins/\emph{name}
\end{alltt}}
in which at least one header file is present.

\texttt{\textbf{[targets]}}\index{[targets]@\texttt{[targets]}}
\index{Makefile keywords@\texttt{Makefile} keywords!\texttt{[targets]}}
is a list of object files
corresponding to the source (\texttt{.c}) files found in
the directories:\index{mkmf@\texttt{mkmf}!default source directories}
\textcolor{black}{\begin{alltt}
  src/plugins/\emph{name}/*.c
  src/vm/intplugins/\emph{name}/*.c
  platforms/Cross/plugins/\emph{name}/*.c
  platforms/unix/plugins/\emph{name}/*.c
\end{alltt}}
where each source file has been stripped of the directory name and had the
`\cmd{.c}' converted into `\texttt{\$o}'.

\texttt{\textbf{[target]}}\index{[target]@\texttt{[target]}}
\index{Makefile keywords@\texttt{Makefile} keywords!\texttt{[target]}}
is the name of the plugin, including the
\texttt{\$a} extension.

\texttt{\textbf{[plibs]}}\index{[plibs]@\texttt{[plibs]}}
\index{Makefile keywords@\texttt{Makefile} keywords!\texttt{[plibs]}}
\label{sec:plibs}
is a list of zero or more libraries on which the plugin depends (as
detected using the macro \texttt{AC\_PLUGIN\_CHECK\_LIB} in the
plugin-specific \cmd{acinclude.m4}).  If the plugin
is being built internally then this list is empty and the required libraries
are included in the final link command.  If the plugin is being built
externally then the plugin itself (a shared object) is linked against
these libraries (via \texttt{[plist]}) rather than with the main VM binary.

(This is to ensure that a missing shared object needed by an external plugin
will only affect the operation of that plugin and not prevent the rest
of the VM from running, which would be the case if the entire VM were
linked against it.)

\texttt{\textbf{[make\_inc]}}\index{[make\_inc]@\texttt{[make\_inc]}}
\index{Makefile keywords@\texttt{Makefile} keywords!\texttt{[make\_inc]}}
is the contents of the
\cmd{Makefile.inc} file in your plugin's \cmd{platdep} directory (or
empty if this file doesn't exist).

\texttt{\textbf{[make\_targets]}}\index{[make\_targets]@\texttt{[make\_targets]}}
\index{Makefile keywords@\texttt{Makefile} keywords!\texttt{[make\_targets]}}
is a list of rules\index{Makefile@\texttt{Makefile}!target rules}
for building the files listed in \texttt{[targets]}.
Each rule\index{plugin!target rules}
looks like this:

\begin{alltt}
        name\$o : \emph{original/source/dir/}name.c
                \$(COMPILE) name\$o \emph{original/source/dir/}name.c
\end{alltt}

\subsubsection{A note about \texttt{\$(COMPILE)} and \texttt{\$(LINK)} commands}

You should \emph{never} pass additional flags to these
commands\index{\$(COMPILE)@\texttt{\$(COMPILE)}}%
\index{\$(LINK)@\texttt{\$(LINK)}}%
\index{Makefile@\texttt{Makefile}!compile/link commands}
explicitly.  This is because you cannot know how they are defined.
(Their definitions depend on whether the plugin is being built
internally or externally~--- and might even change radically in future
releases of Unix Squeak.)

Instead you should pass additional compiler/linker
flags\index{Makefile@\texttt{Makefile}!passing extra flags}
to these commands by setting the following variables in `\cmd{Makefile.inc}' or
`\cmd{Makefile.in}':
\begin{quote}
\index{\$(XCPPFLAGS)@\texttt{\$(XCPPFLAGS)}}
\index{\$(XDEFS)@\texttt{\$(XDEFS)}}
\index{\$(XCFLAGS)@\texttt{\$(XCFLAGS)}}
\index{\$(XLDFLAGS)@\texttt{\$(XLDFLAGS)}}
\begin{tabular}{@{\ttfamily}l@{~~~}l}
XCPPFLAGS	& `\cmd{-I}' flags for \cmd{cpp}		\\
XDEFS		& `\cmd{-D}' flags for \cmd{cpp}		\\
XCFLAGS		& anything to be passed to the compiler	\\
XLDFLAGS	& anything to be passed to the linker	\\
\end{tabular}
\end{quote}
\begin{note}
`\cmd{mkmf}' already uses `\cmd{XINCLUDES}'\index{\$(XINCLUDES)@\texttt{\$(XINCLUDES)}}
\index{Makefile@\texttt{Makefile}!avoiding \texttt{\$(XINCLUDES)}}
to pass the list of
directories containing plugin header files to \cmd{cpp}.  You can
redefine it if you like, but make sure that `\cmd{[includes]}'
appears in its definition (or in the definition of `\cmd{XCPPFLAGS}').
\end{note}

\subsubsection{Specifying additional source directories}

\cmd{mkmf}\index{mkmf@\texttt{mkmf}!additional source directories}
looks for a file in your plugin's
\cmd{platdep} directory called
`\cmd{mkmf.subdirs}'.\index{mkmf.subdirs@\texttt{mkmf.subdirs}}
If this file exists then it should contain a list of directory names relative to
the top-level directory (the one containing the \cmd{src} and
\cmd{platform} directories).  These
directories\index{additional plugin source directories}
will be added to the list of locations searched for
`\cmd{.c}' and `\cmd{.h}' files while
constructing the substitutions for `\texttt{[includes]}',
`\texttt{[targets]}' and `\texttt{[make\_targets]}'.

\subsubsection{Including additional material in the default \texttt{Makefile}}

If the file \cmd{platdep/Makefile.inc}\index{Makefile.inc@\texttt{Makefile.inc}}
exists then \cmd{mkmf} will
substitute its contents into the \cmd{Makefile} in place of the
\texttt{[make\_inc]} keyword.

\begin{note}
\texttt{Makefile.inc}\index{Makefile.inc@\texttt{Makefile.inc}!keyword substitution}
is read into the \texttt{Makefile} under
construction \emph{before} \cmd{mkmf} performs substitutions on the
`\texttt{[keyword]}'s.  In other words, your \cmd{Makefile.inc} can
use the above keywords to include relevant declarations and rules
without worrying about whether the plugin is internal or external.
\end{note}


\subsubsection{Replacing the default \texttt{Makefile} entirely}

If neither of the above are sufficient then you can create a complete
\cmd{Makefile}\index{Makefile@\texttt{Makefile}!replacing}
template called \cmd{platdep/Makefile.in}.\index{Makefile.in@\texttt{Makefile.in}}
\cmd{mkmf} will use this template instead of the default \cmd{Makefile} template
shown earlier, and will perform keyword substitutions on it as
described above to create the final \cmd{Makefile}.  (In other words,
simply copying the default template shown earlier will result in a
\cmd{Makefile} identical to the one that \cmd{mkmf} would have
produced by default.


\subsection{Examples taken from existing plugins}

By way of example we'll look at how two existing plugins specialise
their configuration and \cmd{Makefile}s.

\subsubsection{Configuration}

The \cmd{B3DAcceleratorPlugin} requires OpenGL in order to compile.
The file \cmd{unix/plugins/B3DAcceleratorPlugin/acinclude.m4}\index{acinclude.m4@\texttt{acinclude.m4}!example}
contains a single call to an \cmd{autoconf}-style macro:

\begin{alltt}
    AC\_PLUGIN\_SEARCH\_LIBS(glIsEnabled, GL)
\end{alltt}

This works similarly to the \cmd{autoconf} `\cmd{AC\_SEARCH_LIBS}'
macro: If a library \cmd{libGL.\{a,so\}} (OpenGL) exists and exports
the function \cmd{glIsEnabled()} then `\cmd{-lGL}' is added to the
final VM link command.  Otherwise the plugin is disabled (and a
message warning of the fact is printed).

\begin{note}
There's a bug here.  This should also check for `\cmd{GL\_VERSION\_1\_1}' in headers.
\end{note}


\subsubsection{Customising the \texttt{Makefile}}

The \cmd{Mpeg3Plugin} requires a (modified) \cmd{libmpeg} to be
compiled along with it.  The sources for this library are in (several)
subdirectories of \cmd{Cross/Meg3Plugin} and they require additional
\cmd{cpp} definitions in order to compile correctly.

To cope with the additional directories,
\cmd{unix/plugins/Mpeg3Plugin/mkmf.subdirs}\index{mkmf.subdirs@\texttt{mkmf.subdirs}!example}
simply lists them:

\begin{alltt}
  platforms/Cross/plugins/Mpeg3Plugin/libmpeg
  platforms/Cross/plugins/Mpeg3Plugin/libmpeg/audio
  platforms/Cross/plugins/Mpeg3Plugin/libmpeg/video
\end{alltt}

To cope with the additional \cmd{cpp} definitions, we could have
written a tiny \cmd{Makefile.inc}\index{Makefile.inc@\texttt{Makefile.inc}!example}
containing:

\begin{alltt}
    XDEFS   = -DNOPTHREADS
\end{alltt}

Unfortunately the additional source directories contain various
utility and test programs (which \emph{must not} be built) so we
cannot rely on \cmd{mkmf} generating the correct \cmd{[targets]} list.

Instead we just copy the default \cmd{Makefile} ``template'' (shown
above) as \cmd{Mpeg3Plugin/Makefile.in}
and insert the required list of targets (and \cmd{cpp}
definition) manually.  The end result\index{Makefile.in@\texttt{Makefile.in}!example}
\latexhtml{is shown in Figure~\ref{fig:mf-mpeg}.}{looks like this:}
%begin{latexonly}
\begin{figure}[ht]
%end{latexonly}
\begin{quote}\html{\begin{quote}}
\html{\textcolor{black}}{\begin{alltt}
# Makefile.in for Mpeg3Plugin in Unix Squeak

[make_cfg]
[make_plg]

TARGET  = Mpeg3Plugin\$a

PLUGIN  = Mpeg3Plugin\$o
VIDEO   = getpicture\$o headers\$o idct\$o macroblocks\$o \emph{etc...}
AUDIO   = dct\$o header\$o layer1\$o layer2\$o layer3\$o \emph{etc...}
LIBMPEG = bitstream\$o changesForSqueak\$o libmpeg3\$o \emph{etc...}

OBJS    = \$(PLUGIN) \$(VIDEO) \$(AUDIO) \$(LIBMPEG)

XINCLUDES       = [includes]
XDEFS           = -DNOPTHREADS

\$(TARGET) : \$(OBJS) Makefile
        \$(LINK) \$(TARGET) \$(OBJS)

[make_targets]

.force :
\end{alltt}}\html{\end{quote}}\end{quote}
%begin{latexonly}
\caption{\texttt{unix/plugins/Mpeg3Plugin/Makefile.in}\label{fig:mf-mpeg}}
\end{figure}
%end{latexonly}

\begin{note}
The default `\cmd{[make\_targets]}' will contain additional rules for the
objects that we're trying to avoid building (because it's built from
an exhaustive list of `\cmd{.c}' files in the source directories).  This
does no harm since the offending rules can never be triggered (their
targets are not listed in `\cmd{OBJS}').
\end{note}

\subsection{Coping with VMMaker quirks}

VMMaker\index{VMMaker!missing platform support}
will refuse to compile a plugin if it thinks the plugin
requires platform support.  This is ``all-or-nothing'':\@ if platform
support is required on \emph{one} platform then it is required on
\emph{all} platforms (even if the plugin compiles quite happily
without platform support in Unix).

The easiest way to add ``null'' platform support is to place an empty
`\cmd{Makefile.inc}' in the plugin's \cmd{platdep} directory.  (To see
this in action, look in \cmd{unix/plugins/JPEGReadWriter2Plugin}.)

\subsection{If all else fails}

(Where\index{emergency services}
``all else failing'' is defined as:\@ ``after trying for 20
minutes and still getting nowhere''.)

If you're writing a plugin that needs platform support (beyond dumb
inclusion of a few additional `\cmd{.c}' files) and this document has
been of no help at all (or if you understood it but you're still
suffering from ``all else failing'') then send me
\htmladdnormallink{mail}{mailto:ian.piumarta@inria.fr}
and I'll be
happy to help you with the various platdep files.


\addcontentsline{toc}{section}{Index}

\printindex

\end{document}
